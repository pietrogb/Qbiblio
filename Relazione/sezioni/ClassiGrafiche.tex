\section{Classi Grafiche}{
	La parte View che si è modellata intende permettere all'utente di:

	\begin{figure}[h]
			\begin{center}
				\includegraphics[scale=0.8]{\docsImg Operazioni.jpg}
				\caption{Operazioni previste}
			\end{center}
	\end{figure}
	
	Per questo motivo, è stata creata una classe \textbf{QWidget},
	\subsection{MainWindow}{
		Questa è la classe invocata direttamente dal \textbf{main}; al suo interno è presente:
		\begin{itemize}\itemsep=0.5pt
			\item Un oggetto \textbf{Contenitore<SmartPtr>}, contenente puntatori agli elementi della libreria, dichiarato \underline{privato};
			\item Il costruttore della \textbf{MainWindow}, che crea l'oggetto \textbf{Contenitore} nello heap, fa il setup della finestra e istanzia 
			\item Un metodo \textit{setupView()} che s'occupa di
			\item I seguenti slot:
			\begin{itemize}\itemsep=0.5pt
				\item slotAddItem() per l'inserimento d'un nuovo elemento nella libreria;
				\item slotMostraCatalogoQDialog()  per vedere il contenuto della libreria;
				\item slotTrovaQDialog() per effettuare la ricerca d'un elemento all'interno della libreria
			    \item slotReplaceDVD() per sostituire un film in DVD;
			    \item slotReplaceVHS() per sostituire un film in videocassetta;
			    \item slotReplaceCD()
			    \item slotReplaceLibro()
			    \item slotAggiornaRisultati() 
			\end{itemize}
		\end{itemize}
	}
	
	\subsection{QWidget}{
		Un oggetto di \textbf{QWidget} viene creato dal main senza parent, costruendo così un'\textit{indipendent window}. \\
		Questa classe funge da layout manager, contiene un 
	}
}