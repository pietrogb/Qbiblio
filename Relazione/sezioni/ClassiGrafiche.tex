\section{Classi Grafiche}{
	La parte View che si è modellata intende permettere all'utente di:

	\begin{figure}[h]
			\begin{center}
				\includegraphics[scale=0.8]{\docsImg Operazioni.jpg}
				\caption{Operazioni previste}
			\end{center}
	\end{figure}
	\begin{itemize}\itemsep=0.5pt
		\item Una finestra iniziale da cui sfogliare il catalogo, diviso in sezioni;
		\item La possibilità di guardare il dettaglio di un singolo elemento;
        \item Una finestra da cui inserire un nuovo libro, CD, DVD o VHS;
		\item Una finestra da cui modificare uno degli elementi presenti.
	\end{itemize}
	Per questo motivo, è stata creata una classe \textbf{QWidget},
	\subsection{QWidget}{
		Un oggetto di \textbf{QWidget} viene creato dal main senza parent, costruendo così un'\textit{indipendent window}. \\
		Questa classe funge da layout manager, contiene un 
	}
}