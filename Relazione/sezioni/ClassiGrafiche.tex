\section{Classi Grafiche}{
	La parte View che si è modellata intende permettere all'utente di:

	\begin{figure}[h]
			\begin{center}
				\includegraphics[scale=0.8]{\docsImg Operazioni.jpg}
				\caption{Operazioni previste}
			\end{center}
	\end{figure}
	
	Per questo motivo, è stata creata una classe \textbf{QWidget}.
	\subsection{QWidget}{
		Un oggetto di \textbf{QWidget} viene creato dal main senza parent, costruendo così un'\textit{indipendent window}. \\
		Questa classe funge da layout manager, contiene 5 groupbox al cui interno sono contenute le principali operazioni e funzionalità disponibili, divise nel modo seguente:
	\begin{itemize}\itemsep=0.5pt
		\item \textit{qgb\_menu} rappresenta la groupbox in cui sono racchiusi i pulsanti delle operazioni sulla base di dati della biblioteca;
		\item \textit{qgb\_CD, qgb\_Libri, qgb\_DVD, qgb\_VHS}, in cui sono racchiuse le tabelle che rappresentano - rispettivamente - i CD, Libri, DVD e VHS presenti nella bibioteca;
	\end{itemize}
	Ho scelto di usare le groupboxes per la possibilità di inserire un frame ed un titolo attorno alle tabelle.
	La creazione di una \textbf{MyWidget} provoca l'esecuzione de:
	\begin{itemize}\itemsep=0.5pt
		\item L'impostazione del titolo della finestra, il ridimensionamento della stessa a seconda delle dimensioni dello schermo, lo spostamento al centro dello schermo; il tutto ad opera della funzione \textit{setWidget()}; 
		\item La creazione della \textbf{QGroupBox} \textit{Qgb\_Menu} ad opera della funzione \textit{createMenuGroupBox()}, che connette anche i segnali dei pulsanti ai relativi slot.
		\item La creazione della \textbf{QGroupBox} \textit{Qgb\_CD} ad opera della funzione \textit{createCdGroupBox()};
		\item La creazione della \textbf{QGroupBox} \textit{Qgb\_Libri} ad opera della funzione \textit{createLibriGroupBox()};
		\item La creazione della \textbf{QGroupBox} \textit{Qgb\_DVD}ad opera della funzione \textit{createDvdGroupBox()};
		\item La creazione della \textbf{QGroupBox} \textit{Qgb\_VHS}ad opera della funzione \textit{createVhsGroupBox()}.\\
		Le quattro funzioni che creano le tabelle sono strutturate allo stesso modo, per semplicità verrà illustrata solamente \textit{createCdGroupBox()}: 
		\begin{enumerate}
			\item Il puntatore \textit{qgb\_CD} viene asssociato ad una nuova QGroupBox cui è stato assegnato il titolo: "Catalogo CD";
			\item Viene creata la tabella \textit{tableWidget\_CD} ed impostato il relativo header, composto dai nomi degli attributi che la compongono;
			\item Viene creato un nuovo QVBoxLayout, cui viene aggiunta la widget associata alla tabella creata al punto 2;
			\item Il layout della QGroupBox viene impostato a quello creato in precedenza.
		\end{enumerate}
		La funzione \textit{createMenuGroupBox()} provoca le seguenti azioni:
		\begin{enumerate}
			\item Il puntatore \textit{qgb\_Menu} viene asssociato ad una nuova QGroupBox cui è stato assegnato il titolo: "Operazioni";
			\item Vengono creati i pulsanti \textit{Gestione CD, Gestione DVD, Gestione Libri, Gestione VHS, Trova Elemento, Chiudi} che servono ad aprire le finestre di gestione dei contenuti della libreria e permettono di chiudere il programma.
			\item Viene creato \textit{gridLayout}, dove verranno inseriti i pulsanti che compongono il menù; a questo vengono aggiunti i pulsanti creati in precedenza.
			\item Vengono associati ai pulsanti i relativi slot; verranno illustrati solamente \textit{slotFindItemQDialog()} e \textit{slotInsertCD()}, dato che sono identici quelli di Libro, DVD e VHS.
		\end{enumerate}
	\end{itemize}
	Lo slot \textit{slotInsertCD()} crea un nuovo oggetto della classe \textbf{Gestione\_CD}; questo ha a disposizione una form dove inserire tutti i dati del CD per poi procedere con le 3 operazioni (inserimento, rimozione, cancellazione); sono disponibili anche segnali e slot che sono collegati con la QWidget principale che si occupa di invocare le modifiche sull'oggetto \textbf{Biblio}. \\
	Nel caso dell'inserimento, la pressione del pulsante \underline{Inserisci}, fa emettere il segnale \textit{signalInsertCD()}, collegato allo slot \textit{slotInsertCD()}; questo crea uno \textbf{SmartPtr} con l'oggetto restituito dallo slot \textit{slotNewCD} della classe \textbf{GestioneCD} che prende campo per campo i valori inseriti nella form, istanza un nuovo oggetto CD e lo ritorna. A questo punto viene invocato il metodo \textit{Insert} e \textit{save} sull'oggetto \textbf{Biblio} e vengono aggiornate le tabelle nella QWidget. \\
	I metodi di \textit{slotRemoveCD()}, \textit{slotRemoveCD()}, si differenziano per il messaggio d'errore che viene emesso nel caso non sia presente un CD con i campi dichiarati.\\
	Lo slot \textit{slotFindItemQDialog()} crea un nuovo oggetto della classe TrovaElemento, a cui viene passato l'oggetto \textbf{Biblio}. Ciò provoca l'apertura di una nuova QDialog, in cui si trovano i campi dati su cui è possibile effettuare la ricerca: titolo e genere (presenti in tutti i tipi), ed un campo autore, utilizzato al posto di regista ed artista per le classi \textbf{Libro} e quelle discendenti da \textbf{Film}; sono presenti anche delle checkBox per includere i campi nella ricerca , dei radioButton per precisare il tipo dell'oggetto da cercare ed una TextEdit in cui sono inseriti i risultati .\\
	Il pulsante che avvia la ricerca provoca l'avvio dello slot \textit{slot\_start\_search()} provoca la cancellazione dei risultati della ricerca precedente, cui segue la ricerca all'interno degli elementi memorizzati nella biblioteca: viene individuato il tipo dell'oggetto corrente tramite  dynamic\_cast a cascata, qualora il tipo sia stato incluso nella ricerca, vengono prelevati i valori inseriti nei campi nella form e viene eseguito il confronto con l'oggetto corrente. Se la ricerca ha esito positivo, vengono inseriti su una Qstring tutti i campi dell'oggetto, nell'ordine inverso a quello che verrà mostrato usando il metodo \textit{prepend}; tale Qstring viene poi appesa alla \textit{QtextEdit} dei risultati, che viene aggiornata subito dopo.\\
	}
}