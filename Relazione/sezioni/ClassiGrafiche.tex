\section{Classi Grafiche}{
	La parte View che si è modellata intende permettere all'utente di:

	\begin{figure}[h]
			\begin{center}
				\includegraphics[scale=0.8]{\docsImg Operazioni.jpg}
				\caption{Operazioni previste}
			\end{center}
	\end{figure}
	
	Per questo motivo, è stata creata una classe \textbf{QWidget},
	\subsection{MainWindow}{
		Questa è la classe invocata direttamente dal \textbf{main}; al suo interno è presente:
		\begin{itemize}\itemsep=0.5pt
			\item Un oggetto \textbf{Contenitore}, con gli elementi della libreria;
			\item Un metodo 
		\end{itemize}
	}
	
	\subsection{QWidget}{
		Un oggetto di \textbf{QWidget} viene creato dal main senza parent, costruendo così un'\textit{indipendent window}. \\
		Questa classe funge da layout manager, contiene un 
	}
}