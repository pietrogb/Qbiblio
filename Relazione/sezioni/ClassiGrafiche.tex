\section{Classi Grafiche}{
	La parte View che si è modellata intende permettere all'utente di:

	\begin{figure}[h]
			\begin{center}
				\includegraphics[scale=0.8]{\docsImg Operazioni.jpg}
				\caption{Operazioni previste}
			\end{center}
	\end{figure}
	
	Per questo motivo, è stata creata una classe \textbf{QWidget}.
	\subsection{QWidget}{
		Un oggetto di \textbf{QWidget} viene creato dal main senza parent, costruendo così un'\textit{indipendent window}. \\
		Questa classe funge da layout manager, contiene 5 groupbox al cui interno sono contenute le principali operazioni e funzionalità disponibili, divise nel modo seguente:
	\begin{itemize}\itemsep=0.5pt
		\item \textit{qgb\_menu} rappresenta la groupbox in cui sono racchiusi i pulsanti delle operazioni sulla base di dati della biblioteca;
		\item \textit{qgb\_CD, qgb\_Libri, qgb\_DVD, qgb\_VHS}, in cui sono racchiuse le tabelle che rappresentano - rispettivamente - i CD, Libri, DVD e VHS presenti nella bibioteca;
	\end{itemize}
	Ho scelto di usare le groupboxes per la possibilità di inserire un frame ed un titolo attorno alle tabelle.
	La creazione di una \textbf{MyWidget} provoca l'esecuzione de:
	\begin{itemize}\itemsep=0.5pt
		\item L'impostazione del titolo della finestra, il ridimensionamento della stessa a seconda delle dimensioni dello schermo, lo spostamento al centro dello schermo; il tutto ad opera della funzione \textit{setWidget()}; 
		\item La creazione della \textbf{QGroupBox} \textit{Qgb\_Menu} ad opera della funzione \textit{createMenuGroupBox()}, che connette anche i segnali dei pulsanti ai relativi slot.
		\item La creazione della \textbf{QGroupBox} \textit{Qgb\_CD} ad opera della funzione \textit{createCdGroupBox()};
		\item La creazione della \textbf{QGroupBox} \textit{Qgb\_Libri} ad opera della funzione \textit{createLibriGroupBox()};
		\item La creazione della \textbf{QGroupBox} \textit{Qgb\_DVD}ad opera della funzione \textit{createDvdGroupBox()};
		\item La creazione della \textbf{QGroupBox} \textit{Qgb\_VHS}ad opera della funzione \textit{createVhsGroupBox()}.\\
		Le quattro funzioni che creano le tabelle sono strutturate allo stesso modo, per semplicità verrà illustrata solamente \textit{createCdGroupBox()}: 
		\begin{enumerate}
			\item Il puntatore \textit{qgb_CD} viene asssociato ad una nuova QGroupBox cui è stato assegnato il titolo: "Catalogo CD";
			\item Viene creata la tabella \textit{tableWidget_CD} ed impostato il relativo header, composto dai nomi degli attributi che la compongono;
			\item Viene creato un nuovo QVBoxLayout, cui viene aggiunta la widget associata alla tabella creata al punto 2;
			\item Il layout della QGroupBox viene impostato a quello creato in precedenza.
		\end{enumerate}
		La funzione \textit{createMenuGroupBox()} provoca le seguenti azioni:
		\begin{enumerate}
			\item Il puntatore \textit{qgb_Menu} viene asssociato ad una nuova QGroupBox cui è stato assegnato il titolo: "Operazioni";
			\item Vengono creati i pulsanti \textit{Gestione CD, Gestione DVD, Gestione Libri, Gestione VHS, Trova Elemento, Chiudi} che servono ad aprire le finestre di gestione dei contenuti della libreria e permettono di chiudere il programma.
			\item Viene creato \textit{greidLayout}, dove verranno inseriti i pulsanti che compongono il menù; a questo vengono aggiunti i pulsanti creati in precedenza.
			\item Vengono associati ai pulsanti i relativi slot.
		\end{enumerate}
	\end{itemize}
	}
}