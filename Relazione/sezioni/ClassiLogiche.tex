\section{Classi logiche}{
	La gerarchia sviluppata è composta da 5 classi: 
	
	\begin{figure}[!ht]
		\center
		\includegraphics[width=0.5\textwidth]{\docsImg gerarchia.jpg}
		\caption{Gerarchia delle classi}
	\end{figure} 
	La classe LibraryItem rappresenta un oggetto generico inserito nella libreria, per questo motivo ho scelto di renderla polimorfa ed astratta: non è possibile dichiarare oggetti di questa classe. All'interno di questa classe sono memorizzati gli attributi comuni a tutti gli oggetti della gerarchia: \textit{titolo,genere}, entrambi di tipo QString. \\
	Per questa classe e per le altre classi nella gerarchia verranno resi disponibili:
	\begin{itemize}\itemsep=0.5pt
		\item Un costruttore di defalut ed un costruttore per i campi dati della classe;
		\item Un distruttore virtuale;
		\item Un metodo \textit{clone} che restituisce il puntatore ad un copia all'oggetto su cui viene invocato;
		\item Un operatore di uguaglianza ed uno di disuguaglianza;
		\item Metodi \textbf{get} che restituiscono una copia dei campi dati, per ogni campo dati richiesto.
	\end{itemize}
		Le classi  \textbf{Libro, CD, Film} sono classi derivate direttamente dalla classe base \textbf{Utente}; sono tutte concrete, a differenza di \textbf{Film},  polimorfa ed astratta come \textbf{LibraryItem}. \\
		Le classi \textbf{DVD, VHS} sono classi concrete derivate da \textbf{Film}. \\
		Per quanto riguarda i campi dati:
		\begin{itemize}\itemsep=0.5pt
			\item La classe \textbf{Libro} contiene:
			\begin{itemize}\itemsep=0.5pt
				\item Un \textit{autore}, memorizzato tramite un QString;
				\item Un \textit{anno di uscita}, memorizzato tramite un intero;
				\item Un \textit{editore}, memorizzato tramite un QString.
			\end{itemize}
			\item La classe \textbf{Film} contiene:
			\begin{itemize}\itemsep=0.5pt
				\item Un \textit{regista}, memorizzato tramite un QString;
				\item Una \textit{durata} in minuti, memorizzata tramite un intero;
				\item Una \textit{data d'uscita}, memorizzata tramite un QDate.
				%\item Un insieme di attori \textit{attori}, memorizzato tramite un QVector di QString.
			\end{itemize}
			\item La classe \textbf{CD}  contiene:
			\begin{itemize}\itemsep=0.5pt
				\item Un \textit{artista}, memorizzato tramite un QString;
				\item Un \textit{anno di uscita}, memorizzato tramite un intero;
				\item Un \textit{numero di dischi}, memorizzato tramite un intero.
				%\item Un insieme di \textit{tracce}, memorizzato tramite un QVector di QString;
			\end{itemize}
			\item Le classi \textbf{CD, DVD} sono implementazioni di \textbf{Film}; non contengono ulteriori campi dati. Per ognuna sono resi disponibili i costruttori, un distruttore, il metodo clone, operator == e !=.
			\end{itemize}
		\end{itemize}
		
		Per memorizzare gli oggetti della collezione è stata creata una classe \textbf{Contenitore}. All'interno di questa classe.\\
}