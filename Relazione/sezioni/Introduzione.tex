\section{Introduzione}{
	\subsection{Scopo del documento}{
		Lo scopo di questo documento è di presentare in maniera chiara le scelte architetturali fatte.
	}
	\subsection{Scopo del progetto}{
		Il progetto ha come scopo lo sviluppo di un'applicazione per la gestione di una biblioteca. Gli oggetti che compongono la biblioteca sono inseriti in un database che viene utilizzato attraverso una interfaccia grafica.
		Lo sviluppo è stato fatto utilizzando C++ e Qt. 
	}
	\subsection{Specifiche progettuali}{
		Il progetto è destinato ad immagazzinare i titoli di una biblioteca domestica. Gli oggetti che si possono salvare sono:
		\begin{itemize}\itemsep0.5pt
			\item Libri;
			\item Film, siano questi  VHS o DVD.
			\item CD;
		\end{itemize}
		Le funzionalità principali che si intende modellare sono:
		\begin{itemize}\itemsep0.5pt
			\item Aggiunta di nuovi elementi;
			\item Ricerca tra i titoli presenti;
			\item Cancellazione di titoli, tra quelli presenti.
		\end{itemize}
		I vincoli di progetto sono i seguenti:
		\begin{enumerate}\itemsep0.5pt
			\item Definizione ed utilizzo di una gerarchia G di tipi di altezza >= 1 e larghezza >= 1.
			\item Definizione di un opportuno contenitore C, con relativi iteratori, che permetta inserimenti, rimozioni, modifiche.
			\item Utilizzo del contenitore C per memorizzare oggetti polimorfi della gerarchia G.
			\item Il front-end dell’applicazione deve essere una GUI sviluppata nel framework Qt.
		\end{enumerate}
		Il progetto è stato sviluppato in un sistema XUbuntu; è stato utilizzato l'IDE Qt Creator nella versione 3.0.1, con le librerie alla versione 5.2.1.	\\
		Si è provato il progetto sui computer del Laboratorio Informatico Plesso Paolotti (LabP140 - labP036) dove compila ed esegue correttamente.\\
		Il database che compone l'applicazione viene aperto e salvato usando il formato XML: entrambe le operazioni avvengono in maniera automatica grazie ad un path impostato di default. \\
	
	\begin{figure}[h]
		\begin{center}
			\includegraphics[width=0.4\textwidth]{\docsImg modelview-overview.jpg}
			\caption{Model/View architecture}
		\end{center}
	\end{figure}
		Nello sviluppo è stata curata la separazione tra la parte logica e la parte grafica, senza utilizzare il design pattern MVC. Si e` preferita invece un'architettura Model/View, che rende possibile la separazione del modo in cui sono immagazzinati i dati e la loro presentazione all'utente. %[Questa separazione rende possibili modi diversi di presentare i dati, senza dover cambiare la struttura dati sottostante.]	
	}
}